\section{Conclusion and future works}

\begin{frame}{Contributions}

    \begin{center}
        A top-down approach from a system perspective down to solution implementations.
    \end{center}

    Contributions:
    \begin{itemize}
        \item A unifying definition of Knowledge Graph
        \item An architecture for Knowledge Graph-Based Systems
        \item A framework for Ontology Learning
        \item An OWL Information Retrieval ontology
        \item A study of a text-based compared to a Knowledge Graph-based Information Retrieval system
    \end{itemize}
    
\end{frame}

\begin{frame}{Conclusion}

    We have explored:
    \begin{itemize}
        \item A Knowledge Graph definition incorporating ontologies
        \item A Semantic Web-focused implementation of this Knowledge Graph definition
        \item An OWL Information Retrieval Ontology
        \item Two Knowledge Graph-based Information Retrieval System approaches:
        \begin{itemize}
            \item A real-world use case moving from a text-based to a Knowledge Graph-based Information Retrieval System.
            \item An online OWL reasoning-based Information Retrieval use case.
        \end{itemize}
    \end{itemize}

\end{frame}

\begin{frame}{Future works}

    \begin{itemize}
        \item Knowledge Graph-based Information Retrieval system:
        \begin{itemize}
            \item Expand the Knowledge Graph
            \item Expand the approach to other domains
        \end{itemize}
        \item OWL reasoning-based Information Retrieval system:
        \begin{itemize}
            \item Experiment with a real-world use case at scale
            \item Explore distinguishing between searches with no matching documents and incoherent ones 
        \end{itemize}
        \item Implement an end-to-end Knowledge Graph-Based System architecture use case.
    \end{itemize}

\end{frame}

\begin{frame}{Perspectives: Knowledge Graph}

    \begin{figure} [H]
        \begin{center}
            \includegraphics[scale=0.5]{images/semantic_search_example.pdf} 
            \caption{Extended semantic search example.} 
        \end{center}
    \end{figure}

\end{frame}

\begin{frame}{Scientific productions}

    Peer-reviewed international conference papers:
    \begin{itemize}
        \item Sesbouüé, M., Delestre, N., Kotowicz, J.P., Khudiyev, A., Zanni-Merk, C., 2022. An operational architecture for knowledge graph-based systems. Procedia Computer Science 207, 1667-1676. \href{https://doi.org/10.1016/j.procs.2022.09.224}{https://doi.org/10.1016/j.procs.2022.09.224}. Knowledge-Based and Intelligent Information \& Engineering Systems: Proceedings of the 26th International Conference KES2022.
        \item  Schaeffer, M., Sesboüé, M., Kotowicz, J.P., Delestre, N., Zanni-Merk, C., 2023. Olaf: An ontology learning applied framework. Procedia Computer Science 225, 2106-2115. \\ \href{https://doi.org/10.1016/j.procs.2023.10.201}{https://doi.org/10.1016/j.procs.2023.10.201}, 27th International Conference on Knowledge Based and Intelligent Information and Engineering Sytems (KES 2023)
    \end{itemize}

    Open-source software library:
    \begin{itemize}
        \item Ontology Learning Applied Framework Python library implementation:\\\href{https://wikit-ai.github.io/olaf/}{https://wikit-ai.github.io/olaf/}
    \end{itemize}

\end{frame}