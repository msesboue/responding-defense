\documentclass{beamer}


% \usepackage{beamerthemesplit} // Activate for custom appearance
\usepackage[english]{babel}
\usepackage[T1]{fontenc}
\usepackage[utf8]{inputenc}

\usepackage{photo}
\usetheme{Berlin}
\usefonttheme{serif}
\usefonttheme[18pt]{structuresmallcapsserif}
\useinnertheme{circles}
\beamertemplatenavigationsymbolsempty
\setbeamercovered{transparent}

\usepackage{mathtools}

\usepackage{hyperref}

\usepackage{graphicx}
\graphicspath{{images/}} 
\DeclareGraphicsExtensions{.png,.jpg,.pdf}

\usepackage{multirow}
\usepackage{booktabs}
\usepackage{colortbl}
\usepackage{tabularx}
\usepackage{multirow}
\usepackage{threeparttable}
\usepackage{ragged2e}
\usepackage{makecell}


\definecolor{LITIScolor}{RGB}{1,122,175}
\setbeamercolor{structure}{fg=LITIScolor}
\usebackgroundtemplate{\includegraphics[height=\paperheight]
{images/logoSoftBckGrd.pdf}}

\pgfdeclareimage[height=0.5cm]{logo}{TutellesLITIS}
\logo{\pgfuseimage{logo}}

\title{Knowledge Graph-based System for Technical Document Retrieval}
\subtitle{A deductive reasoning-focused exploration}
\author{Matthias Sesboüé}
\date{September 5, 2024}

\begin{document}
    \frame{
    \centering
    \includegraphics[height = 4cm]{logoLITIS}
    
    \includegraphics[height = 1.3cm]{TutellesLITIS}
    }
    \frame{\titlepage}
    
    \section[Table of content]{}
    \frame
    {
    \frametitle{Table of content}
    \tableofcontents
    %   \begin{columns}[t]
    % \begin{column}{5cm}
    % \tableofcontents[sections={1-4}] \end{column}
    %  \begin{column}{5cm} \tableofcontents[sections={3-4}]
    %  \end{column}
    % \end{columns}
    }
    
    \section{Traceparts}
\begin{frame}{Traceparts}

    \begin{itemize}
        \item Founded in 1990 with activities worldwide
        \item One of the world's leading CAD-content platforms for Engineering, Industrial Equipment and Machine Design: \href{http://traceparts.com/}{traceparts.com}
        \item Digital marketing services for more than 800 customers of all sizes and from all industries.
    \end{itemize}
    
    \begin{center}
        \includegraphics[scale=0.2]{images/traceparts_logo.png}
    \end{center}
    
\end{frame}

\begin{frame}{Traceparts CAD-content plateform}
    
    \begin{center}
         The plateform provides access to over supplier-certified product catalogues with 2D drawings, 3D CAD models and product datasheets.
    \end{center}

    \begin{itemize}
        \item Technical content aimed at an engineering audience from multiple industries
        \item Content available in 25 languages
        \item Users can search using :
        \begin{itemize}
            \item A full text search
            \item A list of catalogues
            \item Different classifications
        \end{itemize}
    \end{itemize}

\end{frame}

\begin{frame}{Corpus}
    
    \begin{itemize}
        \item Over $1.1$ million Document Families
        \item Over $127.8$ millions individual documents
        \item $25$ languages
        \item Documents' texts contain average $50$ characters and $7$ words
        \item Over $210$ thousand tags, amongst which:
        \begin{itemize}
            \item Over $2.5$ thousand suppliers and manufacturers
            \item Over $1.9$ thousand catalogues
            \item Over $208$ thousand categories
        \end{itemize}
    \end{itemize}

    Some text content examples are:
    \begin{itemize}
        \item \emph{DIN 912}
        \item \emph{The P01 to P08 pumps are designed to pump lubricating fluids (oil, diesel oil, etc.). Their flow rate is from 1 to 24 L / min; maximum working pressure 10 bar.} 
    \end{itemize}

\end{frame}

\begin{frame}{User searches}
    
    User text searches:
    \begin{itemize}
        \item are composed of domain-specific keywords, notations, identifiers, and
        acronyms.
        \item contain on average 13 characters separated into 2 words.
        \item can come in any languages
    \end{itemize}

    Some common search examples are:
    \begin{center}
        \emph{motor}, \emph{din 912}, and \emph{ball valve}.
    \end{center}
    
\end{frame}



    % Traceparts presentation
    % Traceparts CAD-content plateform
    % Description des utilisateurs
    % Searches descriptions (+ examples)
    % Description des données (+ examples)
    % Description des possibilité de recherche (text, categories, catalogues)

    \begin{frame}{Information Retrieval}
    
    \begin{figure} [H]
        \begin{center}
            \includegraphics[scale=0.8]{images/ir-system-comps.pdf} 
            \caption{Information Retrieval System overview} 
        \end{center}
    \end{figure}

    \begin{center}
        Traditional approaches leverage statistics about the text corpus. Recent methods implement deep learning models.
    \end{center}
    
\end{frame}

\begin{frame}{BM25}

    \begin{itemize}
        \item The more structured the content is the better.
        \item Systems combines multiple methods balanced with parameters.
        \item Deep Learning methods requires relatively long texts.
    \end{itemize}
    
    BM25 (and its many variants) is:
    \begin{itemize}
        \item based on the Term Frequencies and Inverse Document Frequencies (TF-IDF)
        \item still widely used in practice
        \item computes many statistics offline
    \end{itemize}

    \begin{center}
        Traceparts search system is largely based on a BM25 implementation.
    \end{center}
    
\end{frame}

\begin{frame}{Traceparts search system}
    
    \begin{center}
        A text-based search engine.
    \end{center}
    
    \begin{figure} [H]
        \begin{center}
            \includegraphics[scale=0.7]{images/tp_system.pdf} 
            \caption{Traceparts current system} 
        \end{center}
    \end{figure}

    \begin{center}
        Parts configurations are generated with their text content to be searchable.    
    \end{center}
    
\end{frame}

\begin{frame}{Traceparts search system challenges}
    
    Traceparts search challenges come from:
    \begin{itemize}
        \item Short multilingual texts
        \item Technical texts with many synonyms, acronyms, homonyms, and notations
        \item A large and heterogeneous corpus
        \item Multiple engineering domains coverage
        \item High recall but low precision
    \end{itemize}

\end{frame}
    % Introduce common approches to IR (SOTA IR)
    % TP search system (+ diagram)
    % Why is TP search system not working that well
    % (short texts, mulitlingual text, tehcnical text, acronyms, notations, homonyms, synonyms ...)
    
    \section{Thesis RESPONdING}

\begin{frame}{RESPONdING}
    \begin{center}
        Knowledge Graph-based System for Technical Document Retrieval\\A deductive reasoning-focused exploration
    \end{center}
    
    \begin{itemize}
        \item Research objective: Leveraging domain knowledge to enhance Information Retrieval in a technical context.
        \item Technical content is implicitly structured by domain knowledge.
        \item This knowledge must be explicitly machine readable.
        \item KGs are machine readable structured knowledge artifacts.
    \end{itemize}
    
    \begin{center}
        From a keyword-based search to a concept-based one.
    \end{center}
    
\end{frame}

\begin{frame}{Contributions}

    \begin{center}
        A top-down approach from a system perspective down to solution implementations.
    \end{center}

    Contributions:
    \begin{itemize}
        \item A unifying definition of Knowledge Graph
        \item An architecture for Knowledge Graph-Based Systems
        \item A framework for Ontology Learning
        \item An OWL Information Retrieval ontology
        \item A study of a text-based compared to a Knowledge Graph-based Information Retrieval system
    \end{itemize}
    
\end{frame}

\begin{frame}{Manuscript overview}
    \begin{figure} [H]
        \begin{center}
            \includegraphics[scale=0.4]{images/KGBS-detailed-technos-KG-def-conclusion-simplified.pdf} 
            \caption{Manuscript overview} 
        \end{center}
    \end{figure}
\end{frame}

\begin{frame}{Knowledge Graph vs Ontology}

    \begin{figure} [H]
        \begin{center}
            \includegraphics[scale=0.8]{images/kg-def-simple.pdf} 
            \caption{Knowledge Graph definition} 
        \end{center}
    \end{figure}

\end{frame}

% \begin{frame}{Table of contents ?}

% \end{frame}
    % Our proposition, an approache based-on KG
    % Why this approach makes sense from a SOTA perspective
    % Our contributions
    % We adopted a top down approach from a KG system study down to a KG-based system implementation
    % Manuscript summary Diagram + what we explored in the manuscript + what we are going to touch on in the defense
    % A few words on the notion of KGs ?
    % Introduce the plan

    \section{Experiments}

\begin{frame}{Experiments protocol}

    \begin{figure} [H]
        \begin{center}
            \includegraphics[scale=0.55]{images/tp-search-expe-setting.pdf} 
            \caption{Experiments Protocol.} 
        \end{center}
    \end{figure}

\end{frame}

\begin{frame}{Experiments}

    6 distinct systems built iteratively:
    \begin{itemize}
        \item \emph{Text-based system (baseline)}
        \item \textbf{Concept-based system}
        \item Knowledge Graph-based system
        \item Text-based system with implicit knowledge
        \item Concept-based system with implicit knowledge
        \item \textbf{Knowledge Graph-based system with implicit knowledge}
    \end{itemize}
    
    Systems implementations:
    \begin{itemize}
        \item User search concept matching problem as an information retrieval task.
        \item Leverage user search history as implicit knowledge.
        \item Query concept enrichment as a graph traversal task.
    \end{itemize}

\end{frame}

\begin{frame}{Text-based system (baseline)}

    \begin{figure} [H]
        \begin{center}
            \includegraphics[scale=0.8]{images/tp-expe-text-based-sys.pdf} 
            \caption{Text-based system (baseline)} 
        \end{center}
    \end{figure}

\end{frame}

\begin{frame}{Text-based system (baseline) results}

    \begin{table}[htbp]
        \begin{center}
        \small
        \begin{tabular}{c|cc|}
            \toprule
            \multicolumn{1}{l}{}               & \multicolumn{2}{c}{\textbf{\begin{tabular}[c]{@{}c@{}}Text-based\\ system (baseline)\end{tabular}}} \\ \cmidrule(lr){2-3}
            \multicolumn{1}{c|}{\textbf{@k $\downarrow$}}    & \multicolumn{1}{c}{\textbf{MAP@k}} & \textbf{BM@k} \\ \cmidrule(lr){2-3}
            \multicolumn{1}{c|}{\textbf{@5}}   & \multicolumn{1}{c}{0.061}          & 0.114       \\ 
            \multicolumn{1}{c|}{\textbf{@25}}  & \multicolumn{1}{c}{0.064}          & 0.148       \\ 
            \multicolumn{1}{c|}{\textbf{@50}}  & \multicolumn{1}{c}{0.064}          & 0.157       \\ 
            \multicolumn{1}{c|}{\textbf{@100}} & \multicolumn{1}{c}{0.064}          & 0.161       \\ 
            \multicolumn{1}{c|}{\textbf{@350}} & \multicolumn{1}{c}{0.064}          & 0.164      \\ \bottomrule
        \end{tabular}
        \caption{
            Text-based system (baseline) results for different k values.
        }\label{tab:comp-text-concept-kg}
    \end{center}
    \end{table} 

\end{frame}

\begin{frame}{Concept-based system}

    \begin{figure} [H]
        \begin{center}
            \includegraphics[scale=0.7]{images/tp-expe-concept-based-sys.pdf} 
            \caption{Concept-based system} 
        \end{center}
    \end{figure}

\end{frame}

\begin{frame}{Concept-based system results}

    \begin{table}[htbp]
        \begin{center}
        \small
        \begin{tabular}{c|cc|cc|}
            \toprule
            \multicolumn{1}{l}{}               & \multicolumn{2}{c}{\textbf{\begin{tabular}[c]{@{}c@{}}Text-based\\ system (baseline)\end{tabular}}}  & \multicolumn{2}{c}{\textbf{\begin{tabular}[c]{@{}c@{}}Concept-based\\ system\end{tabular}}} \\ \cmidrule(lr){2-5}
            \multicolumn{1}{c|}{\textbf{@k $\downarrow$}}    & \multicolumn{1}{c}{\textbf{MAP@k}}  & \textbf{BM@k} & \multicolumn{1}{|c}{\textbf{MAP@k}} & \textbf{BM@k} \\ \cmidrule(lr){2-5}
            \multicolumn{1}{c|}{\textbf{@5}}   & \multicolumn{1}{c}{0.061}          & 0.114         & \multicolumn{1}{|c}{0.152} & 0.243      \\ 
            \multicolumn{1}{c|}{\textbf{@25}}  & \multicolumn{1}{c}{0.064}          & 0.148         & \multicolumn{1}{|c}{0.159} & 0.334          \\ 
            \multicolumn{1}{c|}{\textbf{@50}}  & \multicolumn{1}{c}{0.064}          & 0.157         & \multicolumn{1}{|c}{0.160} & 0.371         \\ 
            \multicolumn{1}{c|}{\textbf{@100}} & \multicolumn{1}{c}{0.064}          & 0.161         & \multicolumn{1}{|c}{0.161} & 0.403       \\ 
            \multicolumn{1}{c|}{\textbf{@350}} & \multicolumn{1}{c}{0.064}          & 0.164         & \multicolumn{1}{|c}{0.161} & 0.429      \\ \bottomrule
        \end{tabular}
        \caption{
            Text and concept-based systems results for different k values.
        }\label{tab:comp-text-concept-kg}
    \end{center}
    \end{table} 

\end{frame}

\begin{frame}{Knowledge Graph-based system with implicit knowledge}

    \begin{figure} [H]
        \begin{center}
            \includegraphics[scale=0.4]{images/tp-expe-kg-based-search-hist-sys.pdf} 
            \caption{Knowledge Graph-based system with implicit knowledge} 
        \end{center}
    \end{figure}

\end{frame}

\begin{frame}{Knowledge Graph-based system with implicit knowledge results}

\begin{table}[htbp]
        \begin{center}
        \tiny
        \begin{tabular}{c|cc|cc|cc|}
            \toprule
            \multicolumn{1}{l}{}               & \multicolumn{2}{c}{\textbf{\begin{tabular}[c]{@{}c@{}}Text-based\\ system (baseline)\end{tabular}}}  & \multicolumn{2}{c}{\textbf{\begin{tabular}[c]{@{}c@{}}Concept-based\\ system\end{tabular}}} & \multicolumn{2}{c}{\textbf{\begin{tabular}[c]{@{}c@{}}KG-based system\\with search history\end{tabular}}} \\ \cmidrule(lr){2-7} %\cmidrule(lr){2-4}\cmidrule(lr){5-7}\cmidrule(lr){8-10}
            \multicolumn{1}{c|}{\textbf{@k $\downarrow$}}    & \multicolumn{1}{c}{\textbf{MAP@k}}  & \textbf{BM@k} & \multicolumn{1}{|c}{\textbf{MAP@k}} & \textbf{BM@k}  & \multicolumn{1}{|c}{\textbf{MAP@k}} & \textbf{BM@k} \\ \cmidrule(lr){2-7}
            \multicolumn{1}{c|}{\textbf{@5}}   & \multicolumn{1}{c}{0.061}          & 0.114         & \multicolumn{1}{|c}{0.152} & 0.243 & \multicolumn{1}{|c}{0.115}          & \textbf{0.291}        \\ 
            \multicolumn{1}{c|}{\textbf{@25}}  & \multicolumn{1}{c}{0.064}          & 0.148         & \multicolumn{1}{|c}{0.159} & 0.334 & \multicolumn{1}{|c}{0.122}          & \textbf{0.471}         \\ 
            \multicolumn{1}{c|}{\textbf{@50}}  & \multicolumn{1}{c}{0.064}          & 0.157         & \multicolumn{1}{|c}{0.160} & 0.371 & \multicolumn{1}{|c}{0.123}          & \textbf{0.552}         \\ 
            \multicolumn{1}{c|}{\textbf{@100}} & \multicolumn{1}{c}{0.064}          & 0.161         & \multicolumn{1}{|c}{0.161} & 0.403 & \multicolumn{1}{|c}{0.123}          & \textbf{0.624}         \\ 
            \multicolumn{1}{c|}{\textbf{@350}} & \multicolumn{1}{c}{0.064}          & 0.164         & \multicolumn{1}{|c}{0.161} & 0.429 & \multicolumn{1}{|c}{0.124}          & \textbf{0.715}         \\ \bottomrule
        \end{tabular}
        \caption{
            Text, concept, and KG-based systems results for different k values.
        }\label{tab:comp-text-concept-kg}
    \end{center}
    \end{table} 

\end{frame}

\begin{frame}{Quantitative results}

    \begin{table}[htbp]
        \begin{center}
        \small
        \begin{tabular}{ccc}
            \toprule
            {} &  \textbf{No results}  &  \textbf{\begin{tabular}[c]{@{}c@{}}Less than 400\\ results (non empty)\end{tabular}} \\
            \midrule
            \textbf{\begin{tabular}[c]{@{}c@{}}Text-based\\ system (baseline)\end{tabular}}              &     64.48\% &              35.44\% \\
            \textbf{\begin{tabular}[c]{@{}c@{}}Concept-based\\ system\end{tabular}}                      &     11.43\% &              \textbf{88.36\%} \\
            \textbf{\begin{tabular}[c]{@{}c@{}}KG-based system\\with search history\end{tabular}}      &     \textbf{8.10\%} &       51.59\% \\
            \bottomrule
        \end{tabular}
        \caption{
            Comparing all search systems results set corpus.
        }\label{tab:comp-all-sys-res-list}
    \end{center}
    \end{table}

\end{frame}
    % A first approach, from a text-based system to a KG-based one + Objectif (+ diagrams)
    % Experiment protocol
    % Experiments settings (series of the same diagram with greyed components)
    % Results (Two tables) + Main conclusions
    
    \begin{frame}{Online OWL reasoning-based approach}

    An approach focusing on OWL.
    
    \begin{itemize}
        \item An Information Retrieval ontology.
        \item Push knowledge closer to the data.
        \item Model domain knowledge as linked sets of taxonomies. 
    \end{itemize}

    Competency questions:
    \begin{itemize}
        \item CQ1 What are the categories in the user search?
        \item CQ2 What are the documents relevant to a search?
        \item CQ3 What categories are enabled to refine the search?
    \end{itemize}

\end{frame}

\begin{frame}{Information Retrieval Ontology}

    7 classes:
    \begin{itemize}
        \item \emph{CandidateDocument} subclass of \emph{Document} 
        \item \emph{SelectedCategory} and \emph{EnabledCategory} subclasses of \emph{Category}
        \item \emph{SearchContext} subclass of \emph{Search}
    \end{itemize}

    6 object properties:
    \begin{itemize}
        \item \emph{categorises} inverse of \emph{categorisedBy}
        \item \emph{hasSearchCategory} subproperty of \emph{enablesCategory}
        \item \emph{hasDirectSubcategory} subproperty of \emph{hasSubcategory}
    \end{itemize}

\end{frame}

\begin{frame}{Pizza ontology}

    Pizza ontology:
    \begin{itemize}
        \item Well-knowledge ontology built to introduce RDF/RDFS/OWL with examples (and even SHACL)
        \item Simple ontology with class hierarchies of:
        \begin{itemize}
            \item \emph{Pizza} (\emph{hasTopping}, \emph{hasBase})
            \item \emph{PizzaBase}
            \item \emph{PizzaTopping}
        \end{itemize} 
    \end{itemize}

    \begin{itemize}
        \item We use the Pizza ontology for demonstration in interest of time constraints 
    \end{itemize}

\end{frame}

\begin{frame}{Pizza ontology Knowledge Graph}

        \begin{center}
            \includegraphics[scale=0.65]{images/pizza-demo-kg.pdf} 
        \end{center}

\end{frame}
    % Our second exploratory approach (we try to push as much knowledge as possible down to the KG)
    % A few words on the KG definition
    % Novelty of this approach : OWL reasoning mostly used offline, we propose to use it at runtime (in IR) (Stream reasoning)
    % Pizza ontology as an example because of time constraints
    % Knowledge modelling approach (C-box diagram)
    % IR ontology
    % Align with the KG definition
    % Advantages and limitations

    \section{Conclusion and future works}

\begin{frame}{Conclusion}

    We have explored:
    \begin{itemize}
        \item A Knowledge Graph definition incorporating ontologies
        \item A Semantic Web-focused implementation of this Knowledge Graph definition
        \item An OWL Information Retrieval Ontology
        \item Two Knowledge Graph-based Information Retrieval System approaches:
        \begin{itemize}
            \item A real-world use case moving from a text-based to a Knowledge Graph-based Information Retrieval System.
            \item An online OWL reasoning-based Information Retrieval use case.
        \end{itemize}
    \end{itemize}

\end{frame}

\begin{frame}{Future works}

    \begin{itemize}
        \item Knowledge Graph-based Information Retrieval system:
        \begin{itemize}
            \item Expand the Knowledge Graph
            \item Expand the approach to other domains
        \end{itemize}
        \item OWL reasoning-based Information Retrieval system:
        \begin{itemize}
            \item Experiment with a real-world use case at scale
            \item Explore distinguishing between searches with no matching documents and incoherent ones 
        \end{itemize}
        \item Implement an end-to-end Knowledge Graph-Based System architecture use case.
    \end{itemize}

\end{frame}

\begin{frame}{Perspectives: Knowledge Graph}

    \begin{figure} [H]
        \begin{center}
            \includegraphics[scale=0.5]{images/semantic_search_example.pdf} 
            \caption{Extended semantic search example.} 
        \end{center}
    \end{figure}

\end{frame}
    % We implemented only portions of our KGBS
    % We did not touch on OLAF there but it is continuously developped by Marion
    % Expand the KG for Traceparts (+ diagram dream onto-based system)
    
    \begin{frame}{Thank you!}
    
    \begin{center}
    Thank you for your attention, any questions?
    \end{center}
    
\end{frame}

    % \begin{frame}{Knowledge Graph Based system}
    
    % \begin{figure} [H]
    %     \begin{center}
    %         \includegraphics[scale=0.6]{images/.pdf} 
    %         \caption{Experiments Protocol.} 
    %     \end{center}
    % \end{figure}

\end{frame}

\begin{frame}{Knowledge Graph Based system}
    
    % \begin{figure} [H]
    %     \begin{center}
    %         \includegraphics[scale=0.6]{images/.pdf} 
    %         \caption{Experiments Protocol.} 
    %     \end{center}
    % \end{figure}

\end{frame}

\begin{frame}{Information Retrieval ontology}
    
    \begin{Verbatim}
        :SelectedCategory rdf:type owl:Class ;
                rdfs:subClassOf :Category ,
                                [ rdf:type owl:Restriction ;
                                    owl:onProperty :categorises ;
                                    owl:allValuesFrom :CandidateDocument
                                ] ,
                                [ rdf:type owl:Restriction ;
                                    owl:onProperty :enablesCategory ;
                                    owl:allValuesFrom :EnabledCategory
                                ] ,
                                [ rdf:type owl:Restriction ;
                                    owl:onProperty :hasSubcategory ;
                                    owl:allValuesFrom :SelectedCategory
                                ] .
    
        :SearchContext rdf:type owl:Class ;
                    rdfs:subClassOf :Search ,
                                    [ rdf:type owl:Restriction ;
                                        owl:onProperty :hasSearchCategory ;
                                        owl:allValuesFrom :SelectedCategory
                                    ] .
    \end{Verbatim}

\end{frame}

\end{document}